\documentclass{article}
\title{Parabolic focusing} %\LaTeX is a macro for printing the Latex logo
\author{Daniil A. Fadeev}
\newcommand{\dd}{\partial}
\newcommand{\ff}{\frac}
\newcommand{\ci}{\mathbf{i}}
\newcommand{\mathz}{\ooalign{$z$\cr\hfil\rule[.5ex]{.2em}{.06ex}\hfil\cr}}
\usepackage{amsfonts}
\usepackage{amsmath,esint}
\usepackage[inline]{asymptote}

\begin{document}

\maketitle
This is a technical document to ease calculations of beam parabolic focusing parameters.
Let us start with Shr{\"o}edinger equation
\begin{equation}
\label{shroedinger}
\ci\ff{\dd E}{\dd z}+\ff{D}{r}\ff{\dd}{\dd r}\left(r \ff{\dd E}{\dd r}\right) = 0.
\end{equation}
Solution can be derived in the following form:
\begin{equation}
\label{solution_general}
E = A(z) \exp(-a r^2) \exp(\ci b r^2),
\end{equation}
where $A \in \mathbb{C}$, $a,b \in \mathbb{R}$. By substituting (\ref{solution_general}) into (\ref{shroedinger}):
\begin{equation}
\label{subst}
\ci A' - \ci A a' r^2 - A b' r^2 - 4 D A (a-\ci b) + 4DA(a-\ci b)^2 r^2=0,
\end{equation}
we can derive the following relations:
\begin{equation}
\label{sys1}
a' = -8Dab
\end{equation}
\begin{equation}
\label{sys2}
-b'+4D(a^2-b^2)=0
\end{equation}
\begin{equation}
\label{sys3}
\ci A'-4DA(a-\ci b)=0
\end{equation}
From (\ref{sys1}) and (\ref{sys2}) we have:
\begin{equation}
\label{sys2_1}
-a''a+\ff{3}{2}a'^2=32D^2a^4
\end{equation}
\begin{equation}
\label{sys2_2}
b=-\ff{a'}{8Da}
\end{equation}
The final solution is the following:
\begin{equation}
\label{sol1}
a=\ff{1}{\ff{16D^2}{R_0^2}z^2+R_0^2}
\end{equation}
\begin{equation}
\label{sol2}
b=\ff{4Dz}{R_0^2}\ff{1}{\ff{16D^2}{R_0^2}z^2+R_0^2}
\end{equation}
Inverse formulas read as follows:
\begin{equation}
\label{invsol1}
R_0^2=\ff{a}{a^2+b^2}
\end{equation}
\begin{equation}
\label{invsol2}
z=\ff{1}{4D}\ff{b}{a^2+b^2}
\end{equation}
Let us also derive something to get $R_0$ from $z_1$ and initial radius $R_1$ ($1/a(z_1)$). From (\ref{sol1}) we have:
\[ R_0^2 = \ff{R_1^2}{2}-\sqrt{\ff{R_1^4}{4}-16 D^2 z_1^2} \]
Let us also rewrite for focal diameter:
\[ D_0^2 = \ff{D_1^2}{2}-\sqrt{\ff{D_1^4}{4}- 256 D^2 z_1^2} \]
Here $R_0^2$ is beam radius square ($1/a(z=0)$) at focal point ($z=0$).
Now let us find solution for complex anplitude $A$ which can be rewritten in the following form: $A=|A|\exp(\ci \phi)$. Then from (\ref{sys3}) we have:
\[ \ci A' A* - 4D|A|^2(a-\ci b) = 0 \]
\[ -\ci A*' A - 4D|A|^2(a + \ci b) =0 \]
Then
\[ \left(|A|^2\right)' - 8 D |A|^2 b = 0 \]
Using (\ref{sys1}) we can derive the following:
\[ \ff{\left(|A|^2\right)'}{|A|^2}+\ff{a'}{a} = 0 \]
Then for $|A|^2$:
\[ |A|^2 a = \text{const} \]
and for $\phi$:
\[ \phi'+4Da = 0 \]
\section{FWHM}
Let us also add convenience relations for beam width in FWHM notation.
Let the field amplitude to be set in the form
\[ A(r) \propto \exp\left(-\ff{r^2}{r_{A/e}^2}\right) \]
Then intensity would be
\[ I(r) \propto A^2 \propto \exp\left(-2 \ff{r^2}{r_{A/e}^2}\right) \]
here $r_0$ is field raduis at $1/e$. We want intesity radius at $1/2$ level, so
\[ -2 \ff{r^2}{r_{A/e}^2} = -\ln(2) \]
\[ r_{I/2} = \sqrt{\ff{\ln(2)}{2}} \ r_{A/e} \approx 0.588 \ r_{A/e}  \]
and vice versa
\[ r_{A/e} \approx 1.7 \ r_{I/2}  \]

\end{document}

